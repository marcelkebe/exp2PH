\newpage
\section{Durchführung}
\subsection{Röntgenspektrum}
Zu erzeugung der benöigten Röntgenstrahlung wird eine evakuierte Röntgenröhre verwendet.
Dabei wird eine Kupferanode zur Röntenerzeugung genutzt.
Zur bestimmung des Spektrum der Röntgenstrahlung, verwende man einen Lithumfluoridkristall 
(LiF-Kristall) den man in einem Winkel in den Röntgenstrahl stellt. Den Geiger-Müller-
Zähler stellt man mit einem doppelten Winkel zu dem Kristall ausgerichtet, sodass dieser
die Impulsrate eines bestimmten Braggwinkels misst. Über die Bragg'sche Reflexion (Abschnitt \ref{sub:bragg})
kann nun die Wellenlänge bestimmt werden. Es wird somit die Intensität einer bestimmten Wellenlänge gemessen.
Dafür wird der Bereich von 8° bis 25° in 0,1° Schritten vermessen, der sowohl das
gesamte Bremsrektrum und die charakteristische Strahlung aufnimmt.

Warum dieser Bereich?
\subsection{Transmission}
Will man später die Compton-Welllenlänge bestimmen, so kann dies über die 
Abhängigkeit der Transmission von der Wellenlänge gemacht werden. Dafür wird
ein Aluminium-Absorber verwendet. Dieser wird zunächst vor der Blende der Röntgenröhre 
platziert. Nachdem die Intensität der Braggreflektierten Strahlung in einem Winkelbereich
von 7° bis 10° mit 0,1° Schritten gemessen wurde, wird die Messung wiederholt, dabei 
entfernt man den Absorber aus der Apperatur.
Über eine lineare Transmission-Wellenlängen Abhängigkeit, kann später auf die Wellenlänge
der Compton-Strahlung zurück geschlossen werden.

\subsection{Compton-Wellenlänge}
Der Geiger-Müller-Zähler wird hierfür mit einem Winkel von 90° zum 
Röntgenstrahl platziert. Somit erweden auch nur die Inensitäten der 90° 
gestreuten Phontonen gemessen.
Um nun die Grundintensität $I_0$ der comton gestreuten Phontonen zu messen
verwendet man ein Plexiglas zu Streuung, das mit einem 45° Winkel zum Röntgenstrahl
platziert ist. In gleicher Art misst man die Intensität $I_1$, 
allerdings nun mit dem Absorber der vor die Blende gesetzt wird.
Zum Schluss positioniert man den Absorber vor dem Geiger-Müller-Zähler für
die Intensität $I_2$.

\begin{figure}[H]
    \centering
    \includegraphics[width=0.6\textwidth]{plots/aufbau.jpg}
    \caption{Versuchsaufbau zur Bestimmung der verschiedenen Intensitäten I.\\
        a) zeigt dabei den Aufbau zur Messung von $I_1$, bei welcher der Absorber zwischen 
        Blende und Plexiglas gestellt wird.\\
        b) zeigt dabei den Aufbau zur Messung von $I_2$, bei welcher der Aborber zwischen
        Plexiglas und Geiger-Müller-Zähler gestellt wird.
    }
\end{figure}
\label{sec:Durchfuehrung}
