\newpage
\section{Auswertung}
Die charakteristischen Strahlung/Peaks im Röntgenspektrum sind abbhängig vom 
verwendeten Annodenmaterial Kupfer. Die Linien befinden sich bei \cite{literatur}
\begin{align*}
    K_{\alpha}=8,038\si{keV},\\
    K_{\beta}=8,905\si{keV},
\end{align*}
und somit liegen sie bei den Wellenlängen (Gl. \ref{eqn:energie})
\begin{align*}
    \lambda_{\alpha} =& \frac{hc}{E_{K_{\alpha}}}\approx 1,6\cdot 10^{-10}\si{m},\\
    \lambda_{\beta}  =& \frac{hc}{E_{K_{\beta}}}\approx 1,4 \cdot 10^{-10}\si{m} .
\end{align*}
Dieser Wellenlänge wird nach Gl. \ref{eqn:bragg} und Gl. \ref{eqn:energie} der Braggwinkel $\alpha$ 
mit der Gitterkonstante $d=201,4\cdot 10^{-12}$ von
\begin{align*}
    \alpha_{K_{\alpha}}=&sin^{-1}\left(\frac{hc}{2d \cdot E_{K_{\alpha}}}\cdot\right)\approx 23°,\\
    \alpha_{K_{\beta}}=&sin^{-1}\left(\frac{hc}{2d \cdot E_{K_{\beta}}}\cdot\right)\approx 21°,
\end{align*}
zugeordnet.

\begin{figure}[H]
    \centering
    \includegraphics[width=0.65\textwidth]{plots/welll_int.pdf}
    \caption{Die Impulsrate $I_0$ gegen den Braggwinkel $\alpha$ und die 
    Röntgenwellenlänge $\lambda$ aufgetragen. Es zeigten sie die Peaks der 
    charakteristischen Strahlung, sowie der Berg der Bremsstrahlung, der ab einem 
    Winkel $\alpha=8°$ bis zum ersten Peak verläuft. Die Beschleunigungsspannung 
    beträgt $35$kV.}
    \label{fig:braggw}
\end{figure}

\begin{figure}[H]
    \centering
    \includegraphics[width=0.65\textwidth]{plots/E_int.pdf}
    \caption{Das Energiespektrum der Röntgenstrahlung. Es zeigen sich
    die charakteristischen Röntgenstrahlen bei $K_{\alpha}=8,05$keV und 
    $K_{\beta}=8,92$keV}
    \label{fig:energie}
\end{figure}


Man entnimmt den Messdaten und aus Abb. \ref{fig:energie} und Abb. \ref{fig:braggw} 
folgende charakteristische Strahlungswerte
\begin{align*}
    K_{\alpha}=&8,05keV &\alpha_{K\alpha}=&22,5°& \lambda_{K\alpha}=&154,14\si{pm}\\ 
    K_{\beta}=&8,92keV &\alpha_{K\beta}=&20,2° &\lambda_{K\beta}=&139,09\si{pm}\\
\end{align*}



\subsection{Wellenabhängigkeit der Tansmission}
Um eine Aussage über die Abhängigkeit der Transmission von der Wellenlänge
zu machen, müssen die gemessenen Impulsraten $N_0$ und $N_1$ über Gl. \ref{eqn:totzeit}
mit der Totzeit $\tau=90\mu$s korrigiert werden. Die Transmission ergibt sich dann aus dem Quotienten
von $I_0$ und $I_1$
\begin{equation}
    T_0=\frac{I_1}{I_0}.
    \label{eqn:trans}
\end{equation}
Trägt man nun die aus den Braggwinkel mithilfe der Bragg'schen Reflexion
berechneten Wellenlänge $\lambda$ gegen die Transmission $T$ auf, so folgt
nach linearer Regression über die Geradengleichung
\begin{equation*}
    T=m\cdot \lambda + n,
\end{equation*}
mit der Steigung $m$ und dem y-Achsenabschnitt $n$
\begin{align*}
    m=&(-15.20\pm0.23)\cdot 10^{-9}\si{W\per\m^3},\\ 
    n=&(1.231\pm0.014)\cdot 10^{-12}\si{W\per\m^2},
\end{align*}
die Abhängigkeit $T(\lambda).$
\begin{figure}[H]
    \centering
    \includegraphics[width=0.65\textwidth]{plots/trans.pdf}
    \caption{Gezeigt ist die prozentuale Transmission $T$, die der Wellenlänge $\lambda$
    gegenübergestellt ist. Die lineare Regression zeigt den Zusammenhang $T(\lambda)$.
    Verwendet wurde ein Aluminium Absorber.}
\end{figure}
\label{sub:trans}


\subsection{Die Compton-Wellenlänge}
Bei gleichem Streuwinkel wurden nun die Intensität für compton-gestreute Photonen
$I_{gestreut}$ mit Al-Absorber,nicht gestreute Photonen $I_{ungestreut}$, sowie 
die Intensität $I_0$ ohne Absorber gemessen.
Nun kann über die Transmission und den bereitgestellen Zusammenhang (vgl. Abschnitt \ref{sub:trans})
$T(\lambda)$ die Wellenlänge der Compton-Strahlung bestimmt werden.
Die Poisson verteilten Röntgen-Quanten mit der Messunsicherheit\\
\begin{align*}
    \Delta N= \sqrt{N},
\end{align*}
ergeben für die Impulse
\begin{align*}
    I_0=(2730\pm50)\si{Impulse},\\
    I_{gestreut}=(1024\pm32)\si{Impulse},\\
    I_{ungestreut}=(1180\pm34)\si{Impulse}.
\end{align*}
Es ergibt sich für die Transmission T
\begin{align*}
    T_{ungestreut}=\frac{I_{ungestreut}}{I_{0}}=(0.432\pm0.015)\%,\\
    T_{gestreut}=\frac{I_{gestreut}}{I_{0}}=(0.375\pm0.014)\%.
\end{align*}
Aus dem Zusammenhang $T(\lambda)$ folgt dafür
\begin{align*}
    \lambda_{ungestreut}=(52.6\pm1.6)\si{pm},\\
    \lambda_{gestreut}=(56.3\pm1.5)\si{pm}.
\end{align*}
Es folgt somit für die Compton-Wellenlänge $\lambda_C$ nach \ref{eqn:compton} mit
einem Streuwinke von $90$°
\begin{equation*}
    \lambda_C=\lambda_{gestreut}-\lambda_{ungestreut}=(3.8\pm1.1)\si{pm}.
\end{equation*}
\label{sec:Auswertung}
