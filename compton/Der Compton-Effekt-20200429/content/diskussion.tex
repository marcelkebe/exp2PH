\newpage
\section{Diskussion}
\subsection{Das Röntgenspektrum}
Wie in Abbildung \ref{fig:braggw} und \ref{fig:energie} deutlich gemacht wurde,
befindeten sich charateristischen Punkte des Röntgenspektrums bei,\\
\begin{table}[H]
    \centering
    \begin{tabular}{c |c c c}
        \toprule
        &Theorie \cite{literatur} & Messung & Abweichung $\;/\;$\%\\
        \midrule
        $K_{\alpha}\;/\;$keV &8,038 & 8,05&0,1\\
        $K_{\beta}\;/\;$keV &8,905 &8,92&0,2\\
        Braggwinkel $\alpha_{K \alpha}\;/\;$° &23 & 22,5&2\\
        Braggwinkel $\alpha_{K \beta}\;/\;$° &21 & 20,2&4\\
        $\lambda_{\alpha}\;/\;$pm &160 & 154,14&4\\
        $\lambda_{\beta}\;/\;$pm &140 & 123,09&12\\
        \bottomrule
    \end{tabular}
    \caption{Vergleich zwischen Theorie- und Messdaten bezüglich der
    charakteristischen Strahlung im Röntgenspektrum mit entsprechender Abweichung.}
\end{table}
Es zeigt sich, dass die Abweichungen von den Litaraturwerte klein sind 
und somit zu vernachlässigen sind. Der Bremsberg befindet sich bei $56\si{pm}<\lambda<\lambda_{beta}$ sowie
zwischen den charakteristischen Strahlungen und für $\lambda>\lambda_{\alpha}$.
\subsection{Die Compton Wellenlänge}
Für die theoretische Compton Wellenlänge ergibt sich nach \ref{eqn:compton}
\begin{equation*}
    \lambda_{C,Theorie}=2,4\;\si{pm},
\end{equation*}
und aus den Messdaten nach \ref{eqn:welln}.
\begin{equation*}
    \lambda_C=(3,8\pm1,1)\;\si{pm}
\end{equation*}
Dies entspricht einem relativen Fehler von $(50\pm50)$\%.\\
Der Fehler ist vermutlich auf systematische Fehler zurückzuführen.
Unteranderem kann zusätzliche Strahlung die am Aluminium-Absorber gebeugt werden,
vom Geiger-Müller-Zähler aufgenommen werden.\\

Es wird deutlich das die Compton-Wellenlänge im Vergleich zu der Wellenlänge des 
sichtbaren Lichtes sehr klein ist (Fakor $10^{-5}$). Die Veränderung der Wellenlänge
nach dem elastischen Stoß wäre somit für sichbares Licht kaum verändert und der Compton-Effekt
tritt kaum auf. Zudem handelt es sich bei dem Versuch nicht um freie Elektronen, sondern
um gebundene Elektronen. Das Photon muss somit höhere Energie besitzen als die Bindungsenergie
des Elektrons beträgt.

\subsection{Totzeitkorrektur}
Wie Sinnvoll die Totzeitkorrektur für die Impulsraten $N$ ist, ist fraglich, 
da im gewählten Messbereich, von einem Braggwinkel $7$° bis $10$°, nur kleine Zählraten
auftreten und die Korrektur nur für hohe Zählraten sinnvoll ist. 
