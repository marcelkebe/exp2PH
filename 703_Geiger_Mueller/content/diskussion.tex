\newpage
\section{Diskussion}
Abb. \ref{fig:charakter} zeigt die Charakteristik des Geiger-Müller-Zählrohres.
Für die Eigenschaften des Zählrohres ergeben sich 
\begin{align*}
    \text{Plateaubereich:}380-590\si{V}\\
    \text{Plateauanstieg:} (1.2\pm0.35)\si{\volt^{-1}}\%
\end{align*}
Zudem wird in der Abbildung deutlich, dass die Betriebspannung unterhalb von
$650$V bleiben sollte, um den Entladungsbereich das Zährohres nicht zu erreichen und
diese somit nicht zu zerstören.\\
Die Totzeit
\begin{align*}
    T=(115\pm4)\mu s\\
    T_{Osz}=100\mu s
\end{align*}
weichen 15\% von einander ab. Beide Werte liegen dennoch in einem
typischen Bereich für die Totzeit $T_{Lit}\approx 100\mu s$\cite{wiki} eines Geiger-Müller-Zählrohres.\\

In Abb. \ref{fig:ladung} wird deutlich, dass die freigesetzte Ladung pro einfallendem
Teilchen porportional zur gemessenen Stromstärke $I$ an der Anode ist.
