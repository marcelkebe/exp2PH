\newpage
\section{Diskussion}
\subsection{Emissionspektrum}
Das ermittelte Intensitätsspektrum der Kupfer-Röntgenröhre aus Abb. \ref{fig:spektrums}
befindet sich bei $\Theta=28.2°$.\\
Das Intensitätsmaximum der reflektierten Röntgenstrahlung wird beim Bragg-Winkel relativ
zum LiF-Kristall erwartet. Es befindet sich somit bei dem doppelten Winkel von $\Theta=14°$, also $28$° 
zum Strahl. Somit weicht das gemessene Maximum um $0.2°$ (0.7\%) von der theoretischen Lage ab.
Würde die gemessene Position des Maximums mehr als 3° von der theoretischen Lage abweichen, so
würde sich dies als eine Verschiebung in den nachfolgenden Spektren deutlich machen, da somit auch alle Energie
um einen Faktor verschoben werden.

In Abb. \ref{fig:peaks} zeigt sich das Bremsspektrum über den gesamten Messbereich mit
den zusätlichen Peaks der charakteristischen Strahlung. Die maximale Energie des Bremsberges
kann dabei nicht bestimmt werden, da der Messbereich zu klein ist und somit  $\lambda_{min}$ 
nicht gemessen wird. Nach Gl. \ref{eqn:bragg} wäre der Braggwinkel $\Theta$ zu $\lambda_{min}$ Gl. \ref{eqn:minW}
\begin{align*}
    \lambda_{min}&=35.4\si{pm},\\
    \Theta&=5.04°,
\end{align*}
dieser liegt außerhalb des Messbereichs.\\
Das Auflösungsvermögen $A$ (vgl. Tab. \ref{tab:tabelle1}) benötigt keine Einbeziehung von statistischen
Fehler und repräsentiert die Auflösung gut.\\


In Tabelle \ref{tab:vergleich} wird deutlich, dass die Messungen für die verschiedenen
$E_K$ und Abschirmkonstanten $\sigma_K$ die Theorie gut wiederspiegeln, da die 
Abweichungen vom Literaturwert unterhalb von 5\% liegen.\\
Lediglich die Ryberg-Konstante weicht mehr von dem Literaturwert ab. Dies ist darauf zurückzuführen,
dass die $\sqrt{E_K}$ gegen die Ordnungzahl $Z$ aufgetragen wurde. Ein kleiner Abweichung
könnte beim Auftagen gegen $z_{\text{eff}}$ erziehlt werden.\\
Dennoch ist die Abweichung klein genug um die Theorie zu bestätigen.  
\begin{table}[H]
    \centering
    \begin{tabular}{c | c c c}
        \toprule
        {Messgröße} & {Messwert} & {Literaturwert}& Abweichung$\;/\;$\% \\
        \midrule
        $E_{\text{K}_{\alpha}}\;/\;$ eV & 8050 & 8038& 0.14  \\
        $E_{\text{K}_{\beta}}\;/\;$ eV  & 8920 & 8905& 0.16 \\
        $\sigma_{\text{K1}}$         & 3,32 & 3,31 &0.3   \\
        $\sigma_{\text{K2}}$         & 12.47& 12,36 & 0.9 \\
        $\sigma_{\text{K3}}$         & 22,70 & 21,96 & 3.4 \\
        \midrule
        $E_{\text{K, Zn}}\;/\;$ eV      & 9908  & 9650 & 0.44\\
        $E_{\text{K, Ga}}\;/\;$eV      & 10315  & 10370 & 0.53 \\
        $E_{\text{K, Br}}\;/\;$eV      & 13489 & 13470 & 0.15\\
        $E_{\text{K, Rb}}\;/\;$eV      & 15064 & 15200 & 0.9\\
        $E_{\text{K, Sr}}\;/\;$eV      & 16000 & 16100 & 0.62\\
        $E_{\text{K, Zr}}\;/\;$eV      & 17828 & 17990 & 0.90\\
        $\sigma_{\text{K, Zn}}$  & 3,77  & 3,57  & 0.38\\
        $\sigma_{\text{K, Ga}}$  & 3,70 & 3,62 & 2.1 \\
        $\sigma_{\text{K, Br}}$  & 3,84 & 3,85 & 0.3 \\
        $\sigma_{\text{K, Rb}}$  & 4.11  & 3,95 & 4.1  \\
        $\sigma_{\text{K, Sr}}$  & 4.12  & 4,01 & 2.8 \\
        $\sigma_{\text{K, Zr}}$  & 4.28  & 4,11 & 4.24 \\
        \midrule
        $\text{R}_{\infty}\;/\;$ eV & 12.63$\pm$0.22 & 13,61& 7.1$\pm$1.6 \\
        \bottomrule
    \end{tabular}
    \caption{Alle berechneten/ausgelesenen Energien $E_K$ und die Abschirmkonstanten $sigma_K$
    im Vergleich mit den Literaturwerten \cite{Absorptionskanten} und mit prozentualer Abweichung.}
    \label{tab:vergleich}
\end{table}