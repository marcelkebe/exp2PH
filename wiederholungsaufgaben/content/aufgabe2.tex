\section{Aufgabe 2 - Volumenberechnung}
    Berechnen des Volumens eines Holzylinders mit dem Außenradius $R_a$, Innenradius $R_i$ und
    der Höhe $h$:
    \begin{align*}
        R_{a}=(15 \pm 1) \si{cm},\\
        R_{i}=(10 \pm 1) \si{cm},\\
        h=(20 \pm 1) \si{cm}.\\
    \end{align*}
    Da die Größen fehlerbehaftet sind, erfolgt die allgmeiner Fehlerfortpflanzung nach Gauß
    \begin{equation*}
        \Delta f(x_i)=\sqrt{\sum_{i=0}^n \left(\frac{df}{dx_i}\cdot \Delta x_i \right) ^2},
    \end{equation*}
    und spezifisch für den Holzylinder mit
    \begin{align*}
        \Delta V &= \sqrt{\left(\frac{dV}{dR_{a}}\cdot \Delta R_{a}\right)^2 + \left(\frac{dV}{dR_{i}}\cdot \Delta R_{i}\right)^2 + \left(\frac{dV}{dh}\cdot \Delta h\right)^2},\\\\
        \Delta V &= \sqrt{(2 \pi h R_a \Delta R_a)^2 + (-2 \pi h R_i \Delta R_i)^2 + (\pi (R_a^2-R_i^2) \cdot \Delta h)^2} .
    \end{align*}
    Mit dem Volumen $V$ eines Holzylinders
    \begin{equation*}
        V= \pi h \cdot(R_{a}^2-R_{i}^2),
    \end{equation*}
    ergibt sich schließlich das Gesamtvolumen $V_{ges}$
    \begin{equation*}
        V_{ges}=(2500\pi \pm 732\pi)\; \si{cm^3}.
    \end{equation*}