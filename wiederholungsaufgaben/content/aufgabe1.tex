\section{Aufgabe 1 - Bedeutung der Begriffe}
\subsection{Mittelwert}
    Der Mittelwert $\mu$ eines Datensatzes, ist dessen im durchschnitt angenommener Wert.

    \begin{equation*}
        \mu=\frac{1}{n}\sum_{i=0}^n x_i
    \end{equation*}
\subsection{Standardabweichung}
    Die Standartabweichung $\sigma$ gibt an, wie weit jeder Wert eines Datensatzes im durchschnitt von 
    dem Mittelwert abweicht.\\
    Sie beschribt somit eine Diffuson der Messwerte, um den Mittelwert. Je größer der Wert, desto
    mehr streuen die Werte.
    \begin{equation*}
        \sigma = \sqrt{\frac{\sum_{i=0}^n(x_i-\mu)^2}{n}}
    \end{equation*}
\subsection{Unterscheidung zwischen Streuung der Messwerte und der Fehler des Mittelwerts}
    Sind die gemessenen Größen fehlerbehaftet, so ist es auch der Mittelwert. Man spricht dann vom 
    Fehler des Mittelwerts.\\
