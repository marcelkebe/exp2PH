\section{Aufgabe 1 - Bedeutung der Begriffe}
\subsection{Mittelwert}
    Der Mittelwert $\mu$ eines Datensatzes, ist dessen im durchschnitt angenommener Wert.

    \begin{equation*}
        \mu=\frac{1}{n}\sum_{i=0}^n x_i
    \end{equation*}
\subsection{Standardabweichung}
    Die Standartabweichung $\sigma$ gibt an, wie weit jeder Wert eines Datensatzes im durchschnitt von 
    dem Mittelwert abweicht.\\
    Sie beschreibt somit eine Diffuson der Messwerte, um den Mittelwert. Je größer der Wert, desto
    größer ist die Streuung der Werte um den Mittelwert.
    \begin{equation*}
        \sigma = \sqrt{\frac{\sum_{i=0}^n(x_i-\mu)^2}{n}}
    \end{equation*}
\subsection{Unterscheidung zwischen Streuung der Messwerte und der Fehler des Mittelwerts}
    Mit Streuung der Messwerte ist hiermit die Standardabweichung gemeint, also der Fehler der Einzelmessung.
    Der (mittlere) Fehler des Mittelwertes $\Delta \mu$ unterschiedet sich dabei um den Faktor $1/\sqrt{n}$ von diesem, sodass
    \begin{equation*}
        \Delta \mu = \frac{\sigma}{\sqrt{n}}.
    \end{equation*}
